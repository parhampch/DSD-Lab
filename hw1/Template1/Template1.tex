\documentclass[a4paper,12pt]{article}
\usepackage{HomeWorkTemplate}
\usepackage{circuitikz}
\usepackage[shortlabels]{enumitem}
\usepackage{hyperref}
\usepackage{tikz}
\usepackage{amsmath}
\usepackage{amssymb}
\usepackage{tcolorbox}
\usepackage{xepersian}
\settextfont{XB Niloofar}
\usetikzlibrary{arrows,automata}
\usetikzlibrary{circuits.logic.US}
\usepackage{changepage}
\newcounter{problemcounter}
\newcounter{subproblemcounter}
\setcounter{problemcounter}{1}
\setcounter{subproblemcounter}{1}
\newcommand{\problem}[1]
{
	\subsection*{
		پرسش
		\arabic{problemcounter} 
		\stepcounter{problemcounter}
		\setcounter{subproblemcounter}{1}
		#1
	}
}
\newcommand{\subproblem}{
	\textbf{\harfi{subproblemcounter})}\stepcounter{subproblemcounter}
}


\begin{document}
\handout
{هوش مصنوعی}
{دکتر سیاوش بیات سرمدی}
{نیم‌سال اول 1400\lr{-}1401}
{اطلاعیه}
{پرهام چاوشیان}
{98100118}
 {گزارش آزمایش اول}
برای بررسی بخش‌پذیری بر 3 از قاعده زیر استفاده شده است:
\begin{equation*}
\overline{abcd}\,mod\,3\,=\,(a\,mod\,3)\,+\,(b\,mod\,3)\,+\,(c\,mod\,3)\,+\,(d\,mod\,3)
\end{equation*}
برای استفاده از این قاعده ماژولی با نام
$one\_bcd\_3\_checker$
نوشته شده است که ورودی یه عدد 4بیتی $BCD$ است و در خروجی باقی‌مانده آن عدد بر 3 را می‌دهد. علت 4 بیتی بودن خروجی انجام راحتتر عملیات جمع در ماژول‌های دیگر است. روابط جدول کارنو از جدول صحت و جدول کارنوهای زیر آمده‌اند:\\

برای بررسی بخش‌پذیری بر 11 از قاعده زیر استفاده شده است:
\begin{equation*}
\overline{abcd}\,mod\,11\,=\, (b\,+d\,-\,a\,-\,c)\, mod\,11
\end{equation*}
باتوجه به محدوده اعداد به سادگی می‌توان فهمید که اگر
$(b\,+d\,-\,a\,-\,c)\,=\,-11\,\vee \,0,\vee \,11$
باشد عدد بر 11 بخش پذیر است و در غیر این صورت بخش‌پدیر نیست. به جای تفریق کردن ما مقادیر
$x\,=\,b\,+\,d$
و
$y\,=\,a\,+\,c$
را محاسبه می‌کنیم و سپس به کمک یگ 
\end{document}